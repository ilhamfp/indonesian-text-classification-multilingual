\chapter*{Kata Pengantar}
\addcontentsline{toc}{chapter}{Kata Pengantar}

Puji syukur penulis panjatkan ke hadirat Tuhan Yang Maha Kuasa karena atas berkat dan karunia-Nya, penulis dapat menyelesaikan tugas akhir yang berjudul “Klasifikasi Teks Berbahasa Indonesia Menggunakan \textit{Multilingual Language Model (Studi Kasus: Klasifikasi Ujaran Kebencian dan Analisis Sentimen)}” untuk memenuhi syarat kelulusan tingkat sarjana. Penulis juga ingin mengucapkan terima kasih kepada pihak-pihak yang telah membantu dan mendukung penulis selama pengerjaan tugas akhir ini:

\begin{enumerate}
    \item Ibu Dr. Eng. Ayu Purwarianti, ST.,MT., selaku dosen pembimbing yang telah memberikan arahan, nasehat, dan dukungan selama pengerjaan tugas akhir.
    \item Ibu Fariska Zakhralativa Ruskanda S.T., M.T., dan Ibu Dr. Masayu Leylia Khodra, ST., MT. selaku dosen penguji yang telah memberikan evaluasi dan saran kepada penulis.
    \item Ibu Dessi Puji Lestari S.T.,M.Eng.,Ph.D, Ibu Dr. Fazat Nur Azizah S.T., M.Sc., dan Bapak Nugraha Priya Utama, Ph.D. selaku dosen mata kuliah IF4091 Tugas Akhir I K01 dan IF4092 Tugas Akhir II yang telah memberi arahan selama pelaksanaan tugas akhir ini.
    \item Ibu Dra. Harlili M.Sc. selaku dosen wali yang telah memberikan arahan, nasehat, dan dukungan selama empat tahun berkuliah di program studi Teknik Informatika ITB.
    \item Keluarga penulis yang selalu mendukung dan memotivasi penulis untuk tetap semangat dalam kuliah hingga menyelesaikan tugas akhir.
    \item Seluruh staf pengajar yang belum disebutkan dari program studi Teknik Informatika yang telah membekali penulis dengan ilmu dan wawasan untuk mendukung pengerjaan tugas akhir.
    \item Staf Tata Usaha program studi Teknik Informatika yang telah membantu selama perkuliahan khususnya dalam proses administrasi tugas akhir.
    \item Teman-teman penulis yang telah mendukung serta menemani perjalanan kuliah dan pengerjaan tugas akhir ini.

    
\end{enumerate}
Akhir kata, terima kasih banyak kepada semua pihak yang telah secara langsung maupun tidak langsung membantu penyelesaian tugas akhir ini. Penulis berharap tugas akhir ini dapat bermanfaat bagi para pembaca. Penulis juga menyadari bahwa tugas akhir ini tidaklah sempurna. Oleh karena itu, penulis sangat terbuka terhadap kritik dan saran yang membangun terkait tugas akhir ini.

\begin{flushright} 
    Bandung, 5 Juni 2020 \\
    % \begin{figure}[!h]
    %     \raggedleft
    %     \includegraphics[width=0.2\textwidth]{resources/tandatangan.png}
    % \end{figure}
    \vfill
    Penulis
\end{flushright}
\clearpage
