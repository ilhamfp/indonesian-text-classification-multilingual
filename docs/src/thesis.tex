%--------------------------------------------------------------------%
%
% Berkas utama templat LaTeX.
%
% author Ilham Firadusi Putra
% template dari Petra Barus dan Peb Ruswono Aryan
%
%--------------------------------------------------------------------%
%
% Berkas ini berisi struktur utama dokumen LaTeX yang akan dibuat.
%
%--------------------------------------------------------------------%

\documentclass[12pt, a4paper, onecolumn, oneside, final]{report}

%-------------------------------------------------------------------%
%
% Konfigurasi dokumen LaTeX untuk laporan tesis IF ITB
%
% @author Ilham Firdausi Putra
%
%-------------------------------------------------------------------%
%
% Berkas ini merupakan pembaharuan dari berkas awal milik Petra Novandi dan Steven Lolong
%
%-------------------------------------------------------------------%

% Ukuran kertas
\special{papersize=210mm,297mm}

% Setting margin
\usepackage[top=3cm,bottom=2.5cm,left=4cm,right=2.5cm]{geometry}

\usepackage{mathptmx}

% Format citation
\usepackage[backend=bibtex,citestyle=authoryear,sorting=nyt,firstinits=true]{biblatex}
\renewcommand*{\nameyeardelim}{\addcomma\space}
\renewcommand*\finalnamedelim{\addspace\&\space}

% Anti hyphenation
\tolerance=1
\emergencystretch=\maxdimen
\hyphenpenalty=10000
\hbadness=10000

% Judul bahasa Indonesia
\usepackage[russian, bahasa]{babel}
\usepackage[utf8]{inputenc}
\usepackage{csquotes}
\setquotestyle{english}
\usepackage{graphicx}
\usepackage{titling}
\usepackage{blindtext}
\usepackage{sectsty}
\usepackage{chngcntr}
\usepackage{etoolbox}
\usepackage{hyperref}       % Package untuk link di daftar isi.
\usepackage{titlesec}       % Package Format judul
\usepackage{parskip}

% confusion matrix
\usepackage{multirow}

% package di equation
\usepackage{amsmath}
\usepackage{amsfonts}

% Line satu setengah spasi
\renewcommand{\baselinestretch}{1.5}

% Setting judul chapter
\chapterfont{\centering \large}
\titleformat{\chapter}[display]
  {\large\centering\bfseries}
  {\chaptertitlename\ \thechapter}{0pt}
    {\large\bfseries\uppercase}

\titlespacing*{\chapter}{0pt}{-30pt}{40pt}
\titlespacing*{\section}{0pt}{10pt}{0pt}
\titlespacing*{\subsection}{0pt}{10pt}{0pt}

% Setting besar font section
% \newcommand{\secfnt}{\fontsize{8}{12}}
% \newcommand{\ssecfnt}{\fontsize{8}{12}}

% \titleformat{\section}
% {\normalfont\secfnt\bfseries}{\thesection}{1em}{}

% \titleformat{\subsection}
% {\normalfont\ssecfnt\bfseries}{\thesubsection}{1em}{}
% \titleformat*{\section}{\normalsize\bfseries}
% \titleformat*{\subsection}{\normalsize\bfseries}
% \sectionfont{\fontsize{8}{12}\selectfont}

% Untuk nampilin kode
\usepackage[utf8]{inputenc}
 
\usepackage{listings}
\usepackage{xcolor}
 
\definecolor{codegreen}{rgb}{0,0.6,0}
\definecolor{codegray}{rgb}{0.5,0.5,0.5}
\definecolor{codepurple}{rgb}{0.58,0,0.82}
\definecolor{backcolour}{rgb}{0.95,0.95,0.92}
 
\lstdefinestyle{mystyle}{
    backgroundcolor=\color{backcolour},   
    commentstyle=\color{codegreen},
    keywordstyle=\color{magenta},
    numberstyle=\tiny\color{codegray},
    stringstyle=\color{codepurple},
    basicstyle=\ttfamily\footnotesize,
    breakatwhitespace=false,         
    breaklines=true,                 
    captionpos=b,                    
    keepspaces=true,                 
    numbers=left,                    
    numbersep=5pt,                  
    showspaces=false,                
    showstringspaces=false,
    showtabs=false,                  
    tabsize=2
}
 
\lstset{style=mystyle}

% Setting nomor pada subbsubsubbab
\setcounter{secnumdepth}{3}

\makeatletter

\makeatother

% Counter untuk figure dan table.
\counterwithin{figure}{section}
\counterwithin{table}{section}

% bahasa asing
\usepackage{CJKutf8}

% Spacing pada daftar figure dan table
\makeatletter
     \renewcommand*\l@figure{\@dottedtocline{1}{1em}{3.2em}}
\makeatother

\makeatletter
     \renewcommand*\l@table{\@dottedtocline{1}{1em}{3.2em}}
\makeatother


% Ganti judul bibliography
% \renewcommand\bibname{Daftar Pustaka}


\makeatletter

\makeatother

\bibliography{references}

\begin{document}

    %Basic configuration
    \title{
        Klasifikasi Teks Berbahasa Indonesia Menggunakan \textit{Multilingual Language Model} \\
        (Studi Kasus: Klasifikasi Ujaran Kebencian dan Analisis Sentimen)}
    \date{}
    \author{
        Ilham Firdausi Putra\\
        NIM : 13516140
    }

    \pagenumbering{roman}
    \setcounter{page}{0}

    \clearpage
\pagestyle{empty}

\begin{center}
\smallskip

    \Large \bfseries \MakeUppercase{\thetitle}
    \vfill

    \Large Laporan Tugas Akhir
    \vfill

    \large Disusun sebagai syarat kelulusan tingkat sarjana
    \vfill

    \large Oleh

    \Large \theauthor

    \vfill
    \begin{figure}[h]
        \centering
      	\includegraphics[width=0.2\textwidth]{resources/cover-ganesha.jpg}
    \end{figure}
    \vfill

    \large
    \uppercase{
        Program Studi Teknik Informatika \\
        Sekolah Teknik Elektro dan Informatika \\
        Institut Teknologi Bandung
    }

    Juni 2020

\end{center}

\clearpage

    \clearpage
\pagestyle{empty}
\newgeometry{top=3.5cm,bottom=2.5cm,left=3cm,right=2cm}
\begin{center}
\smallskip

    \Large \bfseries \MakeUppercase{\thetitle}
    \vfill

    \Large Laporan Tugas Akhir
    \vfill

    \large Oleh

    \Large \theauthor

    \large Program Studi Teknik Informatika \\
    \normalsize \normalfont Sekolah Teknik Elektro dan Informatika \\
    Institut Teknologi Bandung \\

    \vfill
    \normalsize \normalfont{
        
    }
    
    Telah disetujui dan disahkan sebagai Laporan Tugas Akhir \\
    di Bandung, pada tanggal 5 Juni 2020.

    \vfill
    \normalsize \normalfont
    Pembimbing\\
    \begin{figure}[!h]
        \centering
        \includegraphics[width=0.2\textwidth]{resources/tandatangan_bu_ayu.png}
    \end{figure}
    \underline{Dr. Eng. Ayu Purwarianti, ST.,MT.} \\
    NIP 19770127 200801 2 011

    % \begin{tabular}{c@{\hskip 0.5in}c}
    %     Pembimbing I, & Pembimbing II \\
    %     & \\
    %     & \\
    %     & \\
    %     & \\
    %     Dr. Eng. Ayu Purwarianti, ST.,MT. & Nama dan Gelar Pembimbing II \\
    %     NIP 19770127 200801 2 011 & NIP 123456789 \\
    % \end{tabular}

\end{center}
\restoregeometry
\clearpage

    \clearpage

\chapter*{Lembar Pernyataan}

Dengan ini saya menyatakan bahwa:

\begin{enumerate}

    \item Pengerjaan dan penulisan Laporan Tugas Akhir ini dilakukan tanpa menggunakan bantuan yang tidak dibenarkan.
    \item Segala bentuk kutipan dan acuan terhadap tulisan orang lain yang digunakan di dalam penyusunan laporan tugas akhir ini telah dituliskan dengan baik dan benar.
    \item Laporan Tugas Akhir ini belum pernah diajukan pada program pendidikan di perguruan tinggi mana pun.

\end{enumerate}

Jika terbukti melanggar hal-hal di atas, saya bersedia dikenakan sanksi sesuai dengan Peraturan Akademik dan Kemahasiswaan Institut Teknologi Bandung bagian Penegakan Norma Akademik dan Kemahasiswaan khususnya Pasal 2.1 dan Pasal 2.2.
\vspace{15mm}



\begin{flushright} 
    Bandung, 22 Juni 2020 \\
    \vskip 0.5in
    % \begin{figure}[!h]
    %     \raggedleft
    %     \includegraphics[width=0.2\textwidth]{resources/tandatangan.png}
    % \end{figure}
    Ilham Firdausi Putra \\
    NIM 13516140
\end{flushright}

\clearpage

    \pagestyle{plain}

    \clearpage
\chapter*{ABSTRAK}
\addcontentsline{toc}{chapter}{Abstrak}
\begin{center} 
    \normalsize \bfseries \MakeUppercase{Klasifikasi Teks Berbahasa Indonesia Menggunakan \textit{Multilingual Language Model (Studi Kasus: Klasifikasi Ujaran Kebencian dan Analisis Sentimen)}} 

    \normalsize \normalfont{Oleh\\
    ILHAM FIRDAUSI PUTRA\\
    NIM : 13516140
    }
\end{center}

%taruh abstrak bahasa indonesia di sini
\blindtext

\textbf{Kata kunci:} \textit{multilingual language lodel}, analisis sentimen, klasifikasi ujaran kebencian
\clearpage
    % \input{chapters/abstract-en}
    \chapter*{Kata Pengantar}
\addcontentsline{toc}{chapter}{Kata Pengantar}

Puji syukur penulis panjatkan ke hadirat Tuhan Yang Maha Kuasa karena atas berkat dan karunia-Nya, penulis dapat menyelesaikan tugas akhir yang berjudul “Klasifikasi Teks Berbahasa Indonesia Menggunakan \textit{Multilingual Language Model (Studi Kasus: Klasifikasi Ujaran Kebencian dan Analisis Sentimen)}” untuk memenuhi syarat kelulusan tingkat sarjana. Penulis juga ingin mengucapkan terima kasih kepada pihak-pihak yang telah membantu dan mendukung penulis selama pengerjaan tugas akhir ini:

\begin{enumerate}
    \item Ibu Dr. Eng. Ayu Purwarianti, ST.,MT., selaku dosen pembimbing yang telah memberikan arahan, nasehat, dan dukungan selama pengerjaan tugas akhir.
    \item Ibu Fariska Zakhralativa Ruskanda S.T., M.T., dan Ibu Dr. Masayu Leylia Khodra, ST., MT. selaku dosen penguji yang telah memberikan evaluasi dan saran kepada penulis.
    \item Ibu Dessi Puji Lestari S.T.,M.Eng.,Ph.D, Ibu Dr. Fazat Nur Azizah S.T., M.Sc., dan Bapak Nugraha Priya Utama, Ph.D. selaku dosen mata kuliah IF4091 Tugas Akhir I K01 dan IF4092 Tugas Akhir II yang telah memberi arahan selama pelaksanaan tugas akhir ini.
    \item Ibu Dra. Harlili M.Sc. selaku dosen wali yang telah memberikan arahan, nasehat, dan dukungan selama empat tahun berkuliah di program studi Teknik Informatika ITB.
    \item Keluarga penulis yang selalu mendukung dan memotivasi penulis untuk tetap semangat dalam kuliah hingga menyelesaikan tugas akhir.
    \item Seluruh staf pengajar yang belum disebutkan dari program studi Teknik Informatika yang telah membekali penulis dengan ilmu dan wawasan untuk mendukung pengerjaan tugas akhir.
    \item Staf Tata Usaha program studi Teknik Informatika yang telah membantu selama perkuliahan khususnya dalam proses administrasi tugas akhir.
    \item Teman-teman penulis yang telah mendukung serta menemani perjalanan kuliah dan pengerjaan tugas akhir ini.

    
\end{enumerate}
Akhir kata, terima kasih banyak kepada semua pihak yang telah secara langsung maupun tidak langsung membantu penyelesaian tugas akhir ini. Penulis berharap tugas akhir ini dapat bermanfaat bagi para pembaca. Penulis juga menyadari bahwa tugas akhir ini tidaklah sempurna. Oleh karena itu, penulis sangat terbuka terhadap kritik dan saran yang membangun terkait tugas akhir ini.

\begin{flushright} 
    Bandung, 5 Juni 2020 \\
    % \begin{figure}[!h]
    %     \raggedleft
    %     \includegraphics[width=0.2\textwidth]{resources/tandatangan.png}
    % \end{figure}
    \vfill
    Penulis
\end{flushright}
\clearpage


    % \titleformat*{\section}{\centering\bfseries\Large\MakeUpperCase}
    \titleformat*{\section}{\centering\bfseries\fontsize{8}{12}\MakeUpperCase}

    \tableofcontents
    \listoffigures
    \listoftables
    % \chapter*{Daftar Istilah}
\clearpage
\begin{center}
    \smallskip
    \large \bfseries{Daftar Istilah}

    \begin{table}[h]
        \begin{tabularx}{\textwidth}{|l|X|}
        \textbf{Dataset}     & Kumpulan data yang digunakan untuk melakukan pelatihan, validasi, maupun evaluasi      \\
        \textbf{Model}       & Representasi matematika yang didapat dari hasil pembelajaran menggunakan data latih               \\
        \textbf{Baseline}    & Performa model atau model yang dijadikan acuan dasar                                              \\
        \textbf{Arsitektur}  & Struktur model yang terdiri dari berbagai macam rule, fungsionalitas, dan implementasi            \\
        \textbf{Transformer} & Arsitektur yang pertama kali dideskripsikan oleh (Vaswani et al., 2017) untuk memodelkan sekuens. \\
        \textbf{LearningRate} & Besar perubahan yang dilakukan ke model pada setiap iterasi pembelajaran \\ 
        \textbf{Callback} & Fungsi yang melekat pada fase pembelajaran model \\
        \textbf{EarlyStopping} & Callback yang akan memberhentikan pembelajaran ketika kondisi yang ditentukan telah dipenuhi \\
        \textbf{ReduceeLrOnPlateau} & Callback yang akan menurunkan besar LearningRate ketika model sudah tidak belajar lagi berdasarkan kondisi ditentukan. \\
        \textbf{Fine-tune} & Proses melatih kembali model ke permasalahan spesifik dari model yang sebelumnya sudah dilatih pada data umum \\
        \end{tabularx}
    \end{table}
\end{center}
\clearpage



    % \titleformat*{\section}{\bfseries\Large}
    \titleformat*{\section}{\bfseries\fontsize{8}{12}}
    \titleformat*{\subsection}{\bfseries\fontsize{8}{12}}
    \pagenumbering{arabic}

    %----------------------------------------------------------------%
    % Konfigurasi Bab
    %----------------------------------------------------------------%
    \setcounter{page}{1}
    \renewcommand{\chaptername}{BAB}
    \renewcommand{\thechapter}{\Roman{chapter}}
    %----------------------------------------------------------------%

    %----------------------------------------------------------------%
    % Dafter Bab
    % Untuk menambahkan daftar bab, buat berkas bab misalnya `chapter-6` di direktori `chapters`, dan masukkan ke sini.
    %----------------------------------------------------------------%
    \chapter{Pendahuluan}

Bab pendahuluan ini menjelaskan tentang landasan pembuatan tugas akhir mengenai klasifikasi teks berbahasa Indonesia menggunakan \textit{multilingual language model}. Bab ini terdiri dari latar belakang, rumusan masalah, tujuan, batasan masalah, metodologi, dan jadwal pelaksanaan tugas akhir.

\section{Latar Belakang}

Permasalahan klasifikasi pembelajaran mesin \textit{supervised} dapat digambarkan sebagai berikut: diberikan satu set label klasifikasi C, dan satu set contoh pelatihan E, yang masing-masing telah diberi salah satu label kelas dari C, sistem harus menggunakan contoh pelatihan E untuk membentuk hipotesis yang dapat digunakan untuk memprediksi label kelas dari contoh baru yang tak digunakan untuk melatih \parencite{mitchell_machine_1997}. Dalam permasalahan klasifikasi teks, data pelatihan E merupakan teks seperti dokumen atau komentar daring pada media sosial. Sedangkan label dapat berupa topik, judul, atau informasi lainnya yang dapat diambil dari teks.

Sebelum sebuah teks dapat diproses, algoritma pembelajaran mesin memerlukan representasi teks dalam bentuk numerik. Untuk hal itu, berkembanglah berbagai teknik untuk merepresentasikan teks sebaik mungkin. Bentuk paling sederhananya adalah pendekatan \textit{one-hot-vector} yang merepresentasikan teks berdasarkan ada atau tidaknya saja. Representasi seperti ini memiliki kekurangan seperti tidak diperhatikannya letak kata dan membesarnya representasi kata seiring membesarnya kosa kata. Kekurangan ini dapat diselesaikan dengan \textit{word embedding} seperti Word2vec \parencite{MikolovWord2vec} yang mempelajari representasi kata sebagai vektor bernilai riil. Tetapi kelemahan pemrosesan teks dengan \textit{word embedding} adalah masih dangkalnya representasi. Representasi \textit{word embedding} tidak dapat menangkap interaksi antar kata di kalimat yang kompleks. Oleh karena itu, berkembanglah \textit{language model} seperti BERT \parencite{Devlin_Chang_Lee_Toutanova_2019} dan XLM \parencite{LampleConneau2019} yang tidak hanya belajar di level kata, tetapi sampai dapat memperhatikan konteks di mana kata tersebut berada. 

Pengembangan representasi teks yang akurat memerlukan banyak data teks. Bahasa Indonesia sebagai bahasa yang ingin diteliti di sini memiliki lebih sedikit data teks dibanding bahasa yang lebih populer seperti bahasa Inggris. Sebagai contoh, bahasa Inggris memiliki dataset seperti \textit{Amazon review}\footnote{\url{http://snap.stanford.edu/data/web-Amazon.html}} atau \textit{Yelp review}\footnote{\url{https://www.yelp.com/dataset/challenge}} yang totalnya hingga jutaan data. Banyaknya data ini mendorong perkembangan yang memungkinkan analisis sentimen mendapatkan akurasi 97.5\%\footnote{\url{https://gluebenchmark.com/leaderboard}} (Diakses pada tanggal 1 Mei 2020) seperti pada \textit{benchmark} GLUE\parencite{GLUE2019}.

Untuk menanggulangi masalah kurangnya data, dapat digunakan teknik yang disebut dengan transfer learning lintas bahasa. Transfer Learning adalah teknik melakukan pembelajaran mesin dari sebuah domain, biasanya yang memiliki lebih banyak data, lalu menggunakan model yang sudah dipelajari untuk menyelesaikan masalah di domain lainnya \parencite{ruder2019transfer}. Penggunaan teknik ini sangat sukses mendorong kemajuan besar di berbagai permasalahan pemrosesan teks alami. Dengan transfer learning lintas bahasa, bahasa yang memiliki sumber daya rendah dapat memanfaatkan sumber daya dari bahasa yang jauh lebih kaya.

Dengan berkembangnya akses masyarakat Indonesia ke internet, semakin banyak data teks yang tersedia secara digital. Data ini penuh informasi dan sangat berguna jika diolah. Bagi pemilik bisnis contohnya, komentar warganet di internet dapat dianalisis sentimennya untuk mengetahui reaksi mereka terhadap sesuatu. Lalu bagi yang memiliki website, mendeteksi pelanggaran dalam percakapan online seperti ujaran kebencian atau kasar secara otomatis dapat sangat membantu. Permasalahan analisis sentimen dan klasifikasi ujaran kebencian \& kasar tersebutlah yang akan menjadi fokus dalam tugas akhir ini.

Analisis sentimen adalah proses pendeteksian dan pengekstraksian informasi subjektif mengenai sentimen dalam sebuah teks. Hal ini dapat dilakukan pada beberapa level ekstraksi yaitu pada level dokumen, kalimat, hingga level spesifik terkait aspek tertentu \parencite{Liu2012}. Pada level paling granular, analisis sentimen dilakukan pada level aspek. Pada level kalimat, sentimen ditentukan untuk setiap kalimat meskipun dalam satu kalimat dapat memiliki lebih dari satu aspek. Dan terakhir pada level dokumen, analisis sentimen dilakukan secara keseluruhan meskipun dalam satu dokumen dapat memiliki lebih dari satu kalimat dan aspek sentimen. Meski analisis sentimen pada level lebih granular dapat memberikan analisa lebih detail, analisis sentimen pada level dokumen masih banyak digunakan untuk mengetahui sentimen secara keseluruhan.

Klasifikasi ujaran kebencian \& kasar adalah proses mengategorikan sebuah teks, biasanya komentar di sosial media atau web, berdasarkan masuk atau tidaknya teks tersebut dalam definisi ujaran yang mengandung kebencian \& kasar. Hal ini dapat dilakukan secara biner dan multi-kelas / multi-label. Pada kasus biner, teks hanya dikategorikan sebagai ujaran yang mengandung kebencian \& kasar atau tidak mengandung sama sekali. Sedangkan pada kasus multi-kelas / multi-label, teks selanjutnya dianalisa untuk mengetahui siapa targetnya atau seberapa parah ujaran kebencian \& kasarnya.

Analisis sentimen dan klasifikasi ujaran kebencian \& kasar dapat dilakukan dengan pendekatan berbasis aturan atau statistik. Dalam sentimen analisisis, pendekatan berbasis aturan seperti VADER \parencite{VADER} memanfaatkan kamus kata sentimen untuk menilai sentimen suatu dokumen. Begitu juga dalam klasifikasi ujaran kebencian \& kasar, \parencite{lexicon_hatespeech_2015} memanfaatkan kamus yang berisi kata-kata negatif dan kebencian. Pada pendekatan berbasis aturan, dokumen direpresentasikan sebagai jumlah kemunculan setiap kata. Sedangkan pendekatan berbasis statistik mencoba mempelajari aturan klasifikasi sentimen dengan teknik-teknik pembelajaran mesin. Pada pendekatan berbasis statistik, teks direpresentasikan dalam bentuk numerik dan selanjutnya diproses menggunakan algoritma pembelajaran mesin. Penelitian pembelajaran mesin teranyar yang dilakukan pada bahasa Indonesia adalah penelitian oleh \parencite{CrisdayantiPurwarianti2019} untuk sentimen analisis dan oleh \parencite{Ibrohim_Budi_2019} untuk ujaran kebencian \& kasar.

Hasil penelitian \parencite{LampleConneau2019} membuktikan efektivitas \textit{transfer learning} lintas bahasa dengan \textit{language model} yang sudah dilatih pada Bahasa Inggris ke bahasa lainnya. Dalam penelitiannya, \textit{language model} dilatih pada berbagai bahasa secara \textit{unsupervised} atau tanpa korpus paralel sama sekali. Hasilnya terbukti efektif dalam berbagai masalah mulai dari translasi mesin, pengembangan language model bahasa yang memiliki data teks sedikit, hingga berbagai tugas klasifikasi. Penelitian ini mencoba menggunakan teknik tersebut untuk meningkatkan performa hasil penelitian \parencite{CrisdayantiPurwarianti2019} mengenai analisis sentimen di bahasa Indonesia dam versi biner penelitian \parencite{Ibrohim_Budi_2019} mengenai ujaran kebencian \& kasar. 

\section{Rumusan Masalah}

Berdasarkan latar belakang yang telah dipaparkan pada sub bab sebelumnya, tugas akhir ini akan fokus dalam mengetahui: 
\begin{enumerate}
	\item Bagaimana pengaruh teknik \textit{transfer learning} lintas bahasa dengan \textit{pretrained language model} dalam permasalahan analisis sentimen dan klasifikasi ujaran kebencian \& kasar bahasa Indonesia?
\end{enumerate}

\section{Tujuan}

Tujuan dari tugas akhir ini adalah sebagai berikut:
\begin{enumerate}
	\item Membangun model analisis sentimen dan klasifikasi ujaran kebencian \& kasar bahasa Indonesia menggunakan fitur dari \textit{language model pretraining} lintas bahasa.
	\item Membandingkan performa analisis sentimen dan klasifikasi ujaran kebencian \& kasar menggunakan \textit{language model pretraining} lintas bahasa dengan yang tanpa \textit{language model pretraining} lintas bahasa.
\end{enumerate}

\section{Batasan Masalah}

Batasan masalah diperlukan untuk membatasi dan menspesifikasi sejauh apa hasil tugas akhir ini akan dibuat. Berikut merupakan batasan masalah untuk tugas akhir ini:
\begin{enumerate}
	\item Bahasa yang akan dijadikan sumber pembelajaran adalah bahasa Inggris dan bahasa Indonesia.
	\item Data teks dan label untuk analisis sentimen menggunakan data dari penelitian sebelumnya pada topik ini oleh \parencite{CrisdayantiPurwarianti2019}.
	\item Data teks dan label untuk klasifikasi ujaran kebencian \& kasar menggunakan data dari penelitian sebelumnya pada topik ini oleh \parencite{Ibrohim_Budi_2019}.
\end{enumerate}

\section{Metodologi}

Berikut metodologi yang digunakan dalam pengembangan tugas akhir ini:
\begin{enumerate}
	\item Studi literatur \\
	Mempelajari berbagai literatur baik berupa buku, tesis, jurnal penelitian, makalah, maupun situs yang memuat informasi yang terkait dengan pembangunan analisis sentimen berbasis deep learning dan \textit{transfer learning} lintas bahasa. Studi literatur berfokus kepada \textit{transfer learning} lintas bahasa, analisis sentimen, deep learning, dan metrik evaluasinya.

	\item Analisis dan perancangan solusi \\
	Pada tahap ini, solusi terbaik akan dipilih dari alternatif solusi berdasarkan kelebihan dan kekurangan masing-masing solusi yang akan diimplementasikan pada analisis sentimen bahasa Indonesia dengan \textit{transfer learning} lintas bahasa. Perancangan solusi mencakup penyusunan arsitektur sistem dan perancangan komponen-komponen sistem yang akan dibangun dengan metode tertentu selama proses pengerjaan tugas akhir ini.

	\item Pengembangan dan evaluasi sistem \\
	Pada tahap ini, dilakukan eksperimen dari hasil analisis dan perancangan solusi, mengevaluasi/menguji  hasil eksperimen, dan mengoptimasi kinerja sistem berdasarkan hasil evaluasi. Keluaran utama dari tahap ini adalah analisis terhadap klasifikasi teks berbahasa Indonesia dengan model lintas bahasa.
\end{enumerate}

\section{Sistematika Pembahasan}
Berikut sistematika dan pembahasan tugas akhir ini:
\begin{enumerate}
	\item \textbf{Bab I Pendahuluan} berisi penjelasan yang melandasi pembuatan tugas akhir ini. Hal itu meliputi penjelasan latar belakang, rumusan masalah, tujuan, batasan masalah, metodologi, dan sistematika penulisan.
	\item \textbf{Bab II Studi Literatur} berisi hasil studi literatur terkait representasi teks dan \textit{language model} lintas bahasa. Hal ini meliputi penjelasan mengenai awal mula representasi teks lintas bahasa, perkembangan representasi lintas bahasa dengan \textit{shared sub-word vocabulary}, perkembangan \textit{language model}, dan penelitian terkaitnya.
	\item \textbf{Bab III Analisis dan Rancangan Analisis Sentimen Teks Berbahasa Indonesia Menggunakan Cross-Lingual Language Model Pretrain} berisi analisis permasalahan, analisis rancangan komponen dataset solusi, dan analisis rancangan komponen klasifikasi solusi.
	\item \textbf{Bab IV Eksperimen dan Pembangunan Sistem} berisi rincian proses mengenai eksperimen yang dilakukan pada tugas akhir beserta evaluasi dan analisis terhadap hasil eksperimen. Bab ini juga menjelaskan lebih rinci lingkungan dan implementasi masing-masing komponen beserta evaluasi terhadap hasil implementasi tersebut.
	\item \textbf{Bab V Kesimpulan dan Saran} berisi kesimpulan terhadap hasil penelitan yang telah dilakukan dalam tugas akhir beserta saran-saran terkait pekerjaan lanjutan yang dapat dijadikan sebagai acuan untuk pengembangan selanjutnya.
\end{enumerate}






    \chapter{Studi Literatur}

Bab ini menjelaskan metrik evaluasi, penelitian terkait, perkembangan representasi teks lintas bahasa mulai dari teknik-teknik terdahulu hingga yang terbaru, \textit{language model}, dan detail mengenai \textit{multilingual language model} mBERT dan XLM-R yang akan digunakan. 

\section{Representasi Teks Lintas Bahasa}
    Salah satu bentuk dari Transfer Learning adalah pembelajaran lintas bahasa \parencite{ruder2019transfer}. Pada pembelajaran lintas bahasa, pembelajaran pertama-tama dilakukan di bahasa lain yang lebih populer dan kemudian digunakan pada bahasa lain yang lebih tidak populer. Untuk dapat melakukan hal ini, diperlukan representasi bahasa pada ruang yang sama seperti dapat dilihat pada ilustrasi di gambar \ref{fig:ilustrasi_embedding}.

    \begin{figure}[ht]
        \centering
        \includegraphics[width=1\textwidth]{resources/luong_et_al_2015.jpg}
        \caption{Ilustrasi ruang \textit{embedding} antara dua bahasa \parencite{Luong_Pham_Manning_2015}}
        \label{fig:ilustrasi_embedding}
    \end{figure}

    Sebelumnya, diperlukan ahli untuk membangun kamus kata-kata dan kalimat secara manual untuk dapat membandingkan kata dari dua bahsa. Seiring dengan perkembangan teknologi, berbagai teknik berkembang untuk dapat melakukan hal ini secara otomatis. Berdasarkan \parencite{Wang_Xie_Xu_Yang_Neubig_Carbonell_2019}, secara garis besar terdapat 2 pendekatan berbeda dalam membangun ruang \textit{embedding} antar bahasa:

    \begin{enumerate}
        \item \textit{Monolingual mapping}: Pada teknik ini, model dilatih secara independen pada bahasa masing-masing. Kemudian mapping antara bahasa dipelajari untuk mendapatkan representasi antar bahasa.

        \item \textit{Joint optimization}: Pada teknik ini, model dilatih pada korpus antar bahasa. Model kemudian mengoptimisasi kombinasi dari monolingual dan cross-lingual loss.
    \end{enumerate}

    Pada \textit{monolingual mapping}, salah satu teknik adalah dengan pertama-tama mempelajari representasi bahasa menggunakan model Skip-gram atau Continuous Bag-of-Words (CBOW) yang didistribusikan diusulkan oleh \parencite{MikolovEstimation}. Model-model ini mempelajari representasi kata menggunakan arsitektur \textit{neural network} sederhana yang bertujuan untuk memprediksi tetangga kata. Karena kesederhanaannya, model Skip-gram dan CBOW dapat dilatih pada sejumlah besar data teks. Pada implementasi paralelnya model ini dapat belajar dari miliaran kata dalam hitungan jam.

    Baru-baru ini ditunjukkan bahwa representasi kata yang didistribusikan secara mengejutkan menangkap banyak keteraturan linguistik, dan ada banyak jenis kesamaan di antara kata-kata yang dapat dinyatakan sebagai terjemahan linear \parencite{MikolovLinguistic2013}. Misalnya, operasi vektor "raja" - "pria" + "wanita" menghasilkan vektor yang dekat dengan "ratu".

    Dua model khusus untuk mempelajari representasi kata yang dapat dilatih secara efisien pada banyak data teks adalah model Skip-gram dan CBOW yang diperkenalkan di \parencite{MikolovEstimation}. Dalam model CBOW, tujuan pelatihannya adalah untuk menggabungkan representasi kata di sekitarnya untuk memprediksi kata di tengah. Sedangkan dalam model Skip-gram, tujuan pelatihan adalah untuk mempelajari representasi kata vektor yang pandai memprediksi konteksnya dalam kalimat yang sama \parencite{MikolovEstimation}. Model arsitektur dari dua metode ini ditunjukkan pada Gambar \ref{fig:ilustrasi_cbow_skip_gram}.

    \begin{figure}[ht]
        \centering
        \includegraphics[width=1\textwidth]{resources/cbow-skip-gram-illustration.png}
        \caption{Ilustrasi model CBOW dan Skip-gram. \parencite{MikolovExploiting}.}
        \label{fig:ilustrasi_cbow_skip_gram}
    \end{figure}
    
    Lebih formal, diberi urutan kata pelatihan \(w1, w2, w3,. . . , wT,\) tujuan dari model Skip-gram adalah untuk memaksimalkan probabilitas log rata-rata

    \begin{equation}
        \frac{1}{T}\sum_{t=1}^{T}\begin{bmatrix}
        \sum_{j=-k}^{K}{\log p(w_{t+j}|w_{t})}
        \end{bmatrix}
        \label{eq:1}
    \end{equation}

    di mana \(k\) adalah ukuran jendela pelatihan. Iterasi penjumlahan di bagian dalam berjalan dari \({-k}\) ke \(k\) untuk menghitung probabilitas log benarnya memprediksi kata \(w_{t+j}\) jika kata di tengah \(w_{t}\). Iterasi penjumlahan di luar mencakup semua kata dalam korpus pelatihan. 

    % TODO: Lanjutin penjelasan Skip-gram dari Mikolov: Expoiting
    %Dalam model Skip-gram, setiap kata w dikaitkan dengan dua vektor parameter yang dapat dipelajari, \(u_{w}\) dan \(v_{w}\). Vektor ini adalah vektor "\textit{input}" dan "\textit{output}" dari \(w\) masing-masing. Peluang untuk memprediksi kata dengan benar dengan kata "wj" didefinisikan sebagai

    Ketika dilatih tentang dataset besar, model Skip-gram atau CBOW ini menangkap banyak informasi semantik. Seperti yang disebutkan sebelumnya, kata-kata yang berkaitan erat memiliki representasi vektor yang serupa, misalnya, sekolah dan universitas, danau, dan sungai. Ini karena sekolah dan universitas muncul dalam konteks yang sama, sehingga selama pelatihan representasi vektor dari kata-kata ini didorong untuk menjadi dekat satu sama lain. Lebih menarik lagi, vektor menangkap hubungan antara konsep melalui operasi linier. Misalnya, \(vektor(Prancis) - vektor(Paris)\) mirip dengan \(vektor(Italia) - vektor(Roma)\).

    Pada gambar \ref{fig:ilustrasi_embedding_inggris_spanyol}, dapat dilihat visualisasi hasil pembelajaran Skip-gram atau CBOW. Gambar \ref{fig:ilustrasi_embedding_inggris_spanyol} memvisualisasikan vektor untuk angka dan hewan dalam bahasa Inggris dan Spanyol, dan dapat dengan mudah dilihat bahwa konsep-konsep ini memiliki susunan geometris yang serupa. Hal ini dikarenakan semua bahasa umum memiliki konsep yang didasarkan pada dunia nyata (seperti kucing adalah binatang yang lebih kecil dari seekor anjing), sering kali ada kesamaan kuat antara ruang vektor. Kesamaan pengaturan geometris dalam ruang vektor adalah alasan utama mengapa metode ini dapat bekerja dengan baik.

    \begin{figure}[ht]
        \centering
        \includegraphics[width=1\textwidth]{resources/ilustration-eng-spn-word.png}
        \caption{Ilustrasi ruang \textit{embedding} antara bahasa Inggris dan Spanyol \parencite{MikolovExploiting}.}
        \label{fig:ilustrasi_embedding_inggris_spanyol}
    \end{figure}

    Dikarenakan miripnya representasi bahasa, jika translasi beberapa objek diketahui, contoh angka satu sampai lima, representasi antar bahasa dapat disesuaikan dengan rotasi, \textit{scaling}, dan translasi untuk mendapatkan tranlasi angka lainnya. Lebih formalnya, misalkan diberi satu set pasangan kata dan representasi vektor yang terkait \begin{math} \begin{Bmatrix} {x_{i}, z_{i}} \end{Bmatrix}_{i=1}^{n} \end{math}, di mana \(x_{i}\in\mathbb{R}^{d_{1}}\) adalah representasi terdistribusi dari kata i dalam bahasa sumber, dan \(z_{i}\in\mathbb{R}^{d_{2}}\) adalah representasi vektor dari terjemahannya. Kemudian dicari matriks transformasi \( W\) sehingga \(W x_{i}\) mendekati \(z_{i}\). Dalam praktiknya, \(W\) dapat dipelajari dengan masalah optimasi berikut

    \begin{equation}
        \min_{W}\sum_{i=1}^{n}\left \| Wx_i-z_i \right \|^2
        \label{eq:2}
    \end{equation}

    yang dapat diselesaikan dengan \textit{stochastic gradient descent} \parencite{MikolovExploiting}.

    Setiap kata baru yang diberikan dan representasi vektor kontinu \(x\) dapat dipetakan ke ruang bahasa lain dengan menghitung \(z = W x\) pada saat prediksi. Kemudian kata yang representasinya paling dekat dengan \(z\) dalam ruang bahasa target dapat ditemukan menggunakan \textit{cosine similarity} sebagai metrik jarak. Terlepas dari kesederhanaannya, metode transformasi linier ini bekerja dengan baik dalam eksperimen yang \parencite{MikolovExploiting} jalankan, lebih baik daripada teknik \textit{nearest neighbour} dan juga pengklasifikasi \textit{neural network}.

    Meski \textit{monolingual mapping} sukses mendapatkan ruang \textit{embedding} antar bahasa, teknik ini mahal dan susah diaplikasikan ke bahasa yang memiliki sumber daya rendah. Untuk dapat mengaplikasikan teknik ini diperlukan kamus kata-kata atau kalimat paralel antar bahasa, hal yang seringkali tidak tersedia pada bahasa bersumber daya rendah. Oleh karena itu, mengaplikasikan teknik ini sangat sulit pada bahasa Indonesia. 

    Kemajuan terbaru datang dari teknik \textit{joint optimization}. Teknik \textit{joint optimization} yang sebelumnya memerlukan korpus paralel atau kamus bilingual (\textit{supervised}) \parencite{Xing_Wang_Liu_Lin}, kini dapat dilakukan tanpa korpus paralel atau kamus bilingual (\textit{unsupervised}) \parencite{Devlin_Chang_Lee_Toutanova_2019} \parencite{LampleConneau2019}. 

% \section{ \textit{Language Model Pretraining} Lintas Bahasa}
\section{Language Model}
    Kemajuan dalam representasi teks teranyar datang dari konsel \textit{language model}. Bab ini akan menjelaskan bagian-bagian utama yang menjadi dasar dari \textit{language model} yang akan digunakan. Hal ini adalah arsitektur \textit{transformer}, \textit{masked language modeling}, dan teknik \textit{pretraining}.

    \subsection{Arsitektur Transformer}

    Berdasarkan \parencite{AttentionVaswani2017}, arsitektur \textit{Transformer} terdiri dari \textit{self-attention} dan \textit{point-wise} yang ditumpuk, dan \textit{fully connected layer} untuk enkoder dan dekoder. Ilustrasi secara keseluruhan arsitektur \textit{Transformer} dapat dilihat pada Gambar \ref{fig:ilustrasi_transformer}.

    \begin{figure}[ht]
        \centering
        \includegraphics[width=0.4\textwidth]{resources/overview-transformer.png}
        \caption{Ilustrasi \textit{transformer} secara keseluruhan \parencite{AttentionVaswani2017}.}
        \label{fig:ilustrasi_transformer}
    \end{figure}

    Fungsi \textit{attention} dapat digambarkan sebagai memetakan kueri dan pasangan \textit{key-value} untuk suatu output, di mana kueri (Q), kunci (K), nilai (V), dan \textit{output} semuanya vektor. \textit{Output} dihitung sebagai \textit{weighted sum} dari nilai-nilai, di mana bobot yang ditetapkan untuk setiap nilai dihitung oleh fungsi kompatibilitas kueri dari kunci yang sesuai.

    Pada penelitian \parencite{AttentionVaswani2017}, tipe \textit{attention} ini disebut "\textit{Scaled Dot-Product Attention}". Input terdiri dari kueri dan kunci dimensi \(d_{k}\), dan nilai dimensi \(d_{v}\). \textit{Dot product} dari kueri dihitung dengan semua kunci, membaginya dengan \(\sqrt{d_{k}}\), dan menerapkan fungsi softmax untuk mendapatkan bobot pada nilai. Ilustrasi \textit{attention} dapat dilihat pada Gambar \ref{fig:ilustrasi_attention}.

    \begin{figure}[ht]
        \centering
        \includegraphics[width=0.2\textwidth]{resources/overview-attention.png}
        \caption{Ilustrasi \textit{attention} secara keseluruhan \parencite{AttentionVaswani2017}.}
        \label{fig:ilustrasi_attention}
    \end{figure}

    Pada prakteknya, hasil keluaran fungsi \textit{attention} pada sebuah set kueri dihitung secara besamaan, dikemas bersama menjadi matriks Q. Kunci dan nilai juga dikemas bersama menjadi matriks K dna V. Kemudian hasil keluaran dapat dihitung dengan:

    \begin{equation}
        Attention(Q,K,V) = softmax(\frac{QK^{T}}{\sqrt{d_k}})V
    \end{equation}

    Pada model XLM yang akan digunakan di tugas akhir ini, digunakan model dengan 12-\textit{layer Transformer}. Arsitektur model ini sama dengan arsitektur model yang digunakan \parencite{LampleConneau2019} untuk melakukan klasifikasi pada data tolak ukur XNLI.

    \subsection{Masked Language Modeling (MLM)}
    Dideskripsikan pertama kali pada penelitian \parencite{Devlin_Chang_Lee_Toutanova_2019}, \textit{Masked Language Modeling} terinspirasi dari \textit{Cloze task} \parencite{Taylor_1953}. Teknik MLM secara acak akan menyembunyikan beberapa kata dari input, dan model memiliki objektif untuk menebak kata asli dari kata yang disembunyikan tadi berdasarkan konteks yang ada di sekelilingnya. Tidak seperti pelatihan \textit{language model} lainnya yang berjalan dari kiri-ke-kanan, pelatihan dengan objektif MLM memungkinkan representasi dari kiri dan kanan mengambil peran. Hal ini memungkinkan pelatihan \textit{Transformer} secara dua arah. Ilustrasi dari \textit{masked language modeling} secara keseluruhan dapat dilihat pada Gambar \ref{fig:ilustrasi_transformer}.

    \begin{figure}[ht]
        \centering
        \includegraphics[width=1\textwidth]{resources/ilustrasi-mlm.png}
        \caption{Ilustrasi \textit{masked language modeling} \parencite{LampleConneau2019}.}
        \label{fig:ilustrasi_mlm}
    \end{figure}

    Sama dengan BERT, persentase dari kata yang akan dipilih untuk disembunyikan pada XLM adalah 15 persen. Setelah sebuah posisi dipilih secara acak, sebuah kata kemudian memiliki 80 persen kemungkinan untuk disembunyikan, 10 persen kemungkinan untuk diganti menjadi sebuah kata acak, dan 10 persen kemungkinan tidak diganti sama sekali.

    \subsection{Teknik Pretraining}
    Model BERT dan XLM merupakan perkembangan dari penelitian oleh \parencite{radford2018improving} dan \parencite{HowardRuder2018} yang meneliti \textit{language modeling} untuk melakukan \textit{pretraining Transformer encoder}. Penelitian-penelitian tersebut sukses membuktikan mangkusnya teknik ini dengan meraih kenaikan performa yang tinggi pada dataset tolak ukur GLUE \parencite{GLUE2019}.

    Teknik \textit{language model pretraining} sukses dikarenakan kemampuannya untuk memanfaatkan data-data teks yang ada dari berbagai sumber tanpa perlu melakukan pelabelan \textit{(unsupervised)}. Pada salah satu model XLM, \textit{pretraining} dilakukan pada data Wikipedia dari 100 bahasa. Melalui pembelajaran dari Wikipedia 100 bahasa ini XLM berhasil mempelajari representasi teks berbagai bahasa.

\section{Multilingual BERT}
    Multilingual BERT adalah salah satu model yang penelitian \parencite{Devlin_Chang_Lee_Toutanova_2019} rilis. Model ini pada dasarnya memiliki arsitektur sama dengan varian BERT-Base yang terdiri dari 12 blok layer Transformer, 768 \textit{hidden size}, 12 \textit{self-attention heads}, dan total 110 juta parameter. Hanya saja model ini tidak dilatih dengan data dari BooksCorpus (800 juta kata) dan Wikipedia bahasa Inggris (2,5 miliar kata). Melainkan, model ini dilatih dengan data Wikipedia dari lebih 104 bahasa.
    
    Meski tidak didesain secara eksplisit untuk memiliki representasi antar bahasa, penelitian \parencite{Pires_Schlinger_Garrette_2019} menunjukan bahwa Multilingual BERT memiliki kemampuan generalisasi antar bahasa yang sangat baik. Pada penelitiannya, \parencite{Pires_Schlinger_Garrette_2019} mendemonstrasikan kemampuan Multilingual BERT pada berbagai permasalahan dan menyiratkan terdapatnya informasi linguistik yang berguna antar bahasa  pada representasi di dalam Multilingual BERT
  
\section{XLM-RoBERTa}
    Meski Multilingual BERT memiliki kemampuan generalisasi antar bahasa, masih terdapat banyak kekurangan yang menghambat kemampuan generalisasi ini maju.
    Penelitian berjudul "RoBERTa: Pendekatan Pretraining BERT yang Lebih Optimal" oleh \parencite{Liu_Ott_Goyal_Du_Joshi_Chen_Levy_Lewis_Zettlemoyer_Stoyanov_2019} menunjukkan bahwa teknik pelatihan dan beberapa pilihan desain dalam pemodelan BERT tidak optimal. Lalu dalam pembangunannya BERT juga tidak didesain dan dioptimisasi secara khusus untuk dapat generalisasi antar bahasa, hal yang \parencite{Conneau_XLMR} teliti dan perbaiki. Sub bab selanjutnya akan membahas beberapa hal tersebut dengan lebih detail.

    \subsection{Desain BERT: Next Sentence Prediction dan Static Masking}
    Penelitian \parencite{Liu_Ott_Goyal_Du_Joshi_Chen_Levy_Lewis_Zettlemoyer_Stoyanov_2019} mengevaluasi kembali dua desain kunci pada pelatihan BERT yaitu fungsi \textit{loss Next Sentence Prediction} dan cara \textit masking pada MLM. Dari hasil evaluasi, diketahui bahwa \textit{Next Sentence Prediction} tidak terlalu membantu dan \textit{Dynamic Masking} lebih baik dibanding \textit{Static Masking}
 
    \subsection{Data}
    XLM-R Dilatih dengan data yang lebih banyak dan lebih beragam. Dapat dilihat pada \ref{fig:data_xlm_r} Jumlah data dalam GiB (skala log) untuk 88 bahasa yang muncul di Wiki-100 corpus yang digunakan untuk mBERT, dan CC-100 yang digunakan untuk XLM-R. CC-100 meningkatkan jumlah data secara signifikan, khususnya untuk bahasa sumber daya rendah.
    \begin{figure}[ht]
        \centering
        \includegraphics[width=1\textwidth]{resources/data_xlm_r.png}
        \caption{Perbedaan data mBERT (Wikipedia) dengan XLM-R (CC) \parencite{Conneau_XLMR}}
        \label{fig:data_xlm_r}
    \end{figure}

    \subsection{Shared sub-word vocabulary}
    Berdasarkan penelitian oleh \parencite{Sennrich_Haddow_Birch_2016}, salah satu inspirasi menggunakan \textit{sub-word} adalah fakta bahwa terjemahan beberapa kata bersifat transparan karena terjemahannya dapat diterjemahkan oleh penerjemah yang kompeten, bahkan jika itu adalah kata yang belum pernah ditemui, berdasarkan terjemahan dari \textit{sub-word} yang dikenal seperti morfemnya atau fonemnya. Kategori kata yang terjemahannya berpotensi transparan meliputi:

    \begin{enumerate}
        \item Entitas bernama (\textit{named entities}). Di antara bahasa yang menggunakan alfabet, nama seringkali dapat disalin dari sumber ke teks target. Transkripsi atau transliterasi mungkin diperlukan pada bahasa dengan huruf atau suku kata berbeda. Contoh: \\
        Barack Obama (Bahasa Inggris dan Jerman) \\
        \foreignlanguage{russian}{Барак Обама} (Bahasa Rusia) \\
        \begin{CJK}{UTF8}{min}
        バラク・オバマ (ba-ra-ku o-ba-ma) (Bahasa Jepang)
        \end{CJK}

        \item Kata-kata serumpun dan pinjaman. Kata serumpun dan kata pinjaman dengan asal yang sama dapat berbeda secara reguler antar bahasa, sehingga aturan penerjemahan tingkat karakter sudah cukup \parencite{Tiedemann2012}. Contoh:  \\
        claustrophobia (Bahasa Inggris) \\
        Klaustrophobie (Bahasa Jerman) \\
        \foreignlanguage{russian}{Клаустрофобия} (Klaustrofobiâ) (Bahasa Rusia)

        \item Kata-kata yang rumit secara morfologis. Kata-kata yang mengandung banyak morfem, misalnya yang dibentuk melalui peracikan, afiksasi, atau infleksi, dapat diterjemahkan dengan menerjemahkan morfem secara terpisah. Contoh:  \\
        solar system (Bahasa Inggris) \\
        Sonnensystem (Sonne + System) (Bahasa Jerman) \\
        Naprendszer (Nap + Rendszer) (Bahasa Hongaria)
    \end{enumerate}

    Penelitian \parencite{Sennrich_Haddow_Birch_2016} membuktikan bahwa segmentasi kata-kata langka ke dalam unit \textit{sub-word} yang sesuai sudah cukup untuk memungkinkan \textit{neural translation network} mempelajari terjemahan yang transparan, dan menggeneralisasikan pengetahuan ini untuk menerjemahkan dan menghasilkan kata-kata yang tidak ditemui sebelumnya. Pada penelitiannya, mereka menggunakan teknik \textit{Byte Pair Encoding} untuk mendapatkan tidak hanya \textit{sub-word}-nya tetapi juga kompresi dari kamus-kamus kata yang ada.

    Teknik \textit{Byte Pair Encoding} (BPE) \parencite{GageBPE1994} adalah teknik kompresi data sederhana yang secara iteratif menggantikan pasangan \textit{byte} yang paling sering muncu; secara berurutan dengan \textit{byte} tunggal yang tidak digunakan. Pada penelitiannya, \parencite{Sennrich_Haddow_Birch_2016} mengadaptasi algoritma \textit{Byte Pair Encoding} untuk segmentasi kata. Alih-alih sering menggabungkan pasangan \textit{byte}, mereka menggabungkan karakter atau urutan karakter.

    Pertama-tama, kosakata simbol diinisialisasi dengan kosakata karakter, dan mewakili setiap kata sebagai urutan karakter, ditambah simbol akhir kata khusus ‘·’, yang memungkinkan hasil terjemahan dikembalikan ke bentuk awal setelah terjemahan. Kemudian semua pasangan simbol dihitung secara iteratif dan untuk setiap pasangan yang paling sering diganti dengan simbol baru. Contoh ('A', 'B') menjadi 'AB'. Setiap operasi penggabungan menghasilkan simbol baru yang mewakili \textit{n-gram} dari karakter. Karakter yang sering muncul (atau seluruh kata) pada akhirnya digabungkan menjadi satu simbol. Ukuran kosa kata simbol akhir sama dengan ukuran kosa kata awal, ditambah jumlah operasi penggabungan --- yang merupakan satu-satunya \textit{hyperparameter} dari algoritma.

    Dapat dilihat contoh sederhana algoritma \textit{Byte Pair Encoding} pada Lampiran \ref{appendix:simple_bpe_algorithm} oleh \parencite{Sennrich_Haddow_Birch_2016}. Pada algoritma ini, operasi \textit{Byte Pair Encoding} belajar dari kamus kata {‘low’, ‘lower’, ‘newest’, ‘widest’}. Pada akhir iterasi, akan diperoleh representasi dari kamus kata dalam bentuk simbol-simbol yang dipelajari. Pada contoh tersebut, kata 'lowest' yang berada diluar kamus kata akan direpresentasikan menjadi gabungan dari simbol 'low' dan 'est'. Di sini dapat dilihat bagaimana \textit{Byte Pair Encoding} dapat membantu merepresentasikan kata yang berada diluar kamus kata. 

\section{Metrik Evaluasi}
% Permasalahan klasifikasi konten negatif di internet merupakan permasalahan yang memiliki kelas tidak seimbang. Dalam permasalahan seperti ini, metrik accuracy dapat menyesatkan dan mengakibatkan paradoks akurasi \parencite{Zhu_Davidson_2007}. 
Metrik yang digunakan mengacu pada penelitian terkait sebelumnya agar dapat dibandingkan. Untuk mengevaluasi performa klasifikasi ujaran kebencian, digunakan akurasi. Sedangkan untuk mengevaluasi performa dari analisis sentimen yang dilakukan pada bahasa Indonesia, digunakan \textit{f1-score}. \textit{f1-score} adalah rata-rata harmonis antara precision dan recall. Berikut definisi dari \textit{precision} (II.1),  \textit{recall} (II.2),  \textit{f1-score} (II.3)  secara matematis:
\begin{equation}
    precision=\frac{True Positive}{True Positive + False Positive}
\end{equation}
\begin{equation}
    recall=\frac{True Positive}{True Positive + False Negative}
\end{equation}
\begin{equation}
    f1-score=2.\: \frac{precision\: .\: recall}{precision+recall}
\end{equation}

Dengan pengelompokan prediksi sesuai dengan \textit{confusion matrix} pada tabel \ref{tab:confusion_matrix}.
\begin{table}[ht]
    \centering
    \begin{tabular}{@{}cc|cc@{}}
    \multicolumn{1}{c}{} &\multicolumn{1}{c}{} &\multicolumn{2}{c}{Prediksi} \\ 
    \multicolumn{1}{c}{} & 
    \multicolumn{1}{c|}{} & 
    \multicolumn{1}{c}{\textit{True}} & 
    \multicolumn{1}{c}{\textit{False}} \\ 
    \cline{2-4}
    \multirow[c]{2}{*}{\rotatebox[origin=tr]{90}{Label}}
    & \textit{True}  & \textit{True Positive} & \textit{False Negative}   \\[1.5ex]
    & \textit{False}  & \textit{False Positive}   & \textit{True Negative} \\ 
    \cline{2-4}
    \end{tabular}
    \caption{Tabel \textit{Confusion matrix}.}
    \label{tab:confusion_matrix}
\end{table}

\section{Penelitian Terkait}
Sebelumnya telah banyak penelitian terkait baik analisis sentimen maupun pembelajan lintas bahasa. \textbf{Pada penelitian \parencite{FarhanKhodra2017}} yang berjudul “\textit{Sentiment-specific word embedding for Indonesian sentiment analysis}”, berbagai bentuk representasi teks dibandingkan dan dievaluasi menggunakan dataset ulasan TripAdvisor. Hasil dari penelitian ini dapat dilihat pada Tabel \ref{tab:FarhanKhodra2017}.

\begin{table}[]
    \centering
    \caption{Hasil eksperimen (F1-score) data ulasan TripAdvisor \parencite{FarhanKhodra2017}.}
    \begin{tabular}{|l|l|l|}
    \hline
    \multicolumn{1}{|c|}{\multirow{2}{*}{\textbf{Word Embedding}}} & \multicolumn{2}{c|}{\textbf{Hasil Evaluasi}}                                                    \\ \cline{2-3} 
    \multicolumn{1}{|c|}{}                                         & \multicolumn{1}{c|}{\textbf{10-fold cross validation}} & \multicolumn{1}{c|}{\textbf{Data Uji}} \\ \hline
    Bag of words                                                   & 0.8345                                                 & 0.8232                                 \\ \hline
    TF-IDF                                                         & \textbf{0.8492}                                        & \textbf{0.8521}                        \\ \hline
    Word2Vec                                                       & 0.7204                                                 & 0.7219                                 \\ \hline
    SSWE dari W2V                                                  & 0.7311                                                 & 0.7224                                 \\ \hline
    SSWE                                                           & 0.7623                                                 & 0.7687                                 \\ \hline
    \end{tabular}
    \label{tab:FarhanKhodra2017}
\end{table} 

\textbf{Pada penelitian \parencite{CrisdayantiPurwarianti2019}} yang berjudul “\textit{Improving Bi-LSTM Performance for Indonesian Sentiment Analysis Using Paragraph Vector}”, dilakukan perbandingan berbagai representasi dokumen dan topologi \textit{neural network}. Hasil terbaik didapatkan oleh model yang dilatih pada representasi kata dengan TF-IDF dan representasi paragraf dengan Doc2vec. Model Bi-LSTM yang dilatih berhasil mendapatkan F1-Score sebesar 0.9369 pada data sentimen dari berbagai media sosial (Twitter, Zomato, TripAdvisor, Facebook, Instagram, Qraved). Hasil dari penelitian ini dapat dilihat pada Tabel \ref{tab:CrisdayantiPurwarianti2019}.

\begin{table}[]
    \centering
    \caption{Hasil evaluasi berbagai model dan representasi dokumen pada \parencite{CrisdayantiPurwarianti2019}.}
    \begin{tabular}{|l|l|l|l|}
    \hline
    \multicolumn{1}{|c|}{\textbf{Model}} & \multicolumn{1}{c|}{\textbf{Precision}} & \multicolumn{1}{c|}{\textbf{Recall}} & \textbf{F1-Score} \\ \hline
    SVM (TF-IDF)                         & 0.7977                                  & 0.8878                               & 0.8658            \\ \hline
    Bi-LSTM (WE)                         & 0.9166                                  & 0.9126                               & 0.9125            \\ \hline
    Bi-LSTM (PV+WE)                      & \textbf{0.9384}                         & \textbf{0.9369}                      & \textbf{0.9369}   \\ \hline
    \end{tabular}
    \label{tab:CrisdayantiPurwarianti2019}
\end{table}


\textbf{Pada penelitian \parencite{LampleConneau2019}} yang berjudul “\textit{Cross-lingual Language Model Pretraining}”, model berarsitektur \textit{transformer} dilatih dengan MLM dan \textit{shared sub-word vocabulary} pada data Wikipedia. Kemudian keluaran dari \textit{layer} terakhir model ini menjadi masukan ke \textit{layer} linier terakhir. Model mereka berhasil memecahkan beberapa state-of-the-art pada  dataset tolak ukur klasifikasi pembelajaran lintas bahasa XNLI \parencite{Conneau_Rinott_Lample_Williams_Bowman_Schwenk_Stoyanov_2018}. Hasil dari penelitian ini dapat dilihat pada Tabel \ref{tab:LampleConneau2019}. Hasil yang ditampilkan hanya 4 dari 15 bahasa dan hanya hasil XLM dengan pendekatan \textit{unsupervised}.

\begin{table}[]
    \centering
    \caption{Hasil pada uji akurasi di dataset tolak ukur klasifikasi pembelajaran lintas bahasa XNLI \parencite{LampleConneau2019}.}
    \begin{tabular}{|l|l|l|l|l|}
    \hline
    \multicolumn{1}{|c|}{\textbf{Evaluasi \textit{Sentence Encoder} Lintas Bahasa}} & \multicolumn{1}{c|}{\textbf{en}} & \multicolumn{1}{c|}{\textbf{fr}} & \textbf{es}   & \textbf{de}   \\ \hline
    \parencite{LamplePhrase2018}                            & 73.7                             & 67.7                             & 68.7          & 67.7          \\ \hline
    \parencite{Devlin_Chang_Lee_Toutanova_2019}         & 81.4                             & -                                & 74.3          & 70.5          \\ \hline
    \parencite{Artetxe_Schwenk_2019}                      & 73.9                             & 71.9                             & 72.9          & 72.6          \\ \hline
    \parencite{LampleConneau2019}                           & \textbf{83.2}                    & \textbf{76.5}                    & \textbf{76.3} & \textbf{74.2} \\ \hline
    \end{tabular}
    \label{tab:LampleConneau2019}
\end{table}

\textbf{Pada penelitian \parencite{Ibrohim_Budi_2019}} yang berjudul "\textit{Multi-label Hate Speech and Abusive Language Detection in Indonesian Twitter}", dilakukan klasifikasi multi-label ujaran kebencian \& kasar. Terdapat dua skenario eksperimen yang dicoba. Pada skenario pertama, klasifikasi dilakukan hingga ke target, kategori, dan levelnya. Hasilnya dapat dilihat pada Tabel \ref{tab:Ibrohim_Budi_2019_1}. Pada skenario kedua, klasifikasi hanya dilakukan sampai menentukan kategori teks ujaran kebencian atau kasar saja. Hasil skenario kedua dapat dilihat pada Tabel \ref{tab:Ibrohim_Budi_2019_2}

\begin{table}[]
    \centering
    \caption{Hasil skenario pertama klasifikasi ujaran kebencian \& kasar pada \parencite{Ibrohim_Budi_2019}.}
    \begin{tabular}{|l|l|l|}
    \hline
    \textbf{Tipe Fitur} & \textbf{Fitur Terbaik Berdasarkan Rata-rata Akurasi} & \textbf{Rata-rata Akurasi (\%)} \\ \hline
    word n-gram              & word unigram + bigram + trigram                 & 59.44                          \\ \hline
    character n-gram         & character quadgrams                             & 52.55                          \\ \hline
    ortography               & question mark                                   & 44.44                          \\ \hline
    lexicon                  & negative sentiment                              & 44.45                          \\ \hline
    \end{tabular}
    \label{tab:Ibrohim_Budi_2019_1}
\end{table}

\begin{table}[]
    \centering
    \caption{Hasil skenario pertama kedua ujaran kebencian \& kasar pada \parencite{Ibrohim_Budi_2019}.}
    \begin{tabular}{|l|l|l|}
    \hline
    \textbf{Tipe Fitur} & \textbf{Fitur Terbaik Berdasarkan Rata-rata Akurasi} & \textbf{Rata-rata Akurasi (\%)} \\ \hline
    word n-gram              & word unigram                                    & 73.53                          \\ \hline
    character n-gram         & character quadgrams                             & 72.44                          \\ \hline
    ortography               & exclamation mark                                & 45.27                          \\ \hline
    lexicon                  & positive sentiment + abusive lexicon            & 52.10                          \\ \hline
    \end{tabular}
    \label{tab:Ibrohim_Budi_2019_2}
\end{table}

% [\textbf{TODO: } Tabel hasil \parencite{LampleConneau2019} pada XNLI]

% \textbf{Pada penelitian \parencite{Lai_Oguz_Yang_Stoyanov_2019}} yang berjudul “\textit{Bridging the Domain Gap in Cross-Lingual Document Classification}”, model tidak hanya dilatih pada representasi kalimat dari XLM untuk dapat mempelajari representasi bahasa, tetapi juga dilakukan augmentasi data untuk mempelajari domain dari data sumber dan target yang digunakan. Hasilnya berhasil menunjukan bahwa aspek domain penting diperhatikan dalam membangun model klasifikasi pembelajaran lintas bahasa.

% [\textbf{TODO: } Tabel hasil \parencite{Lai_Oguz_Yang_Stoyanov_2019}]


% \begin{table}[]
%     \centering
%     \begin{tabular}{|l|l|l|l|l|l|l|l|l|l|l|l|l|l|l|l|}
%     \hline
%     \multicolumn{1}{|c|}{\textbf{Evaluasi Sentence Encoder Lintas Bahasa}} & \multicolumn{1}{c|}{\textbf{en}} & \multicolumn{1}{c|}{\textbf{fr}} & \textbf{es}   & \textbf{de}   & \textbf{el}   & \textbf{bg}   & \textbf{ru}   & \textbf{tr}   & \textbf{ar}   & \textbf{vi}   & \textbf{th}   & \textbf{zh}   & \textbf{hi}   & \textbf{sw}  & \textbf{ur}   \\ \hline
%     \parencite{LamplePhrase2018}                            & 73.7                             & 67.7                             & 68.7          & 67.7          & 68.9          & 67.9          & 65.4          & 64.2          & 64.8          & 66.4          & 64.1          & 65.8          & 64.1          & 55.7         & 58.4          \\ \hline
%     \parencite{Devlin_Chang_Lee_Toutanova_2019}         & 81.4                             & -                                & 74.3          & 70.5          & -             & -             & -             & -             & 62.1          & -             & -             & 63.8          & -             & -            & 58.3          \\ \hline
%     \parencite{Artetxe_Schwenk_2019}                           & 73.9                             & 71.9                             & 72.9          & 72.6          & \textbf{73.1} & \textbf{74.2} & 71.5          & \textbf{69.7} & \textbf{71.4} & \textbf{72.0} & 69.2          & 71.4          & 65.5          & 62.2         & 61.0          \\ \hline
%     \parencite{LampleConneau2019}                           & \textbf{83.2}                    & \textbf{76.5}                    & \textbf{76.3} & \textbf{74.2} & \textbf{73.1} & 74.0          & \textbf{73.1} & 67.8          & 68.5          & 71.2          & \textbf{69.2} & \textbf{71.9} & \textbf{65.7} & \textbf{64.} & \textbf{63.4} \\ \hline
%     \end{tabular}
%     \caption{Hasil pada uji akurasi di dataset tolak ukur klasifikasi pembelajaran lintas bahasa XNLI}
%     \label{tab:LampleConneau2019}
% \end{table}
    \chapter{Analisis dan Rancangan Klasifikasi Teks Berbahasa Indonesia Menggunakan \textit{Multi-lingual Language Model}}

Bab ini berisi analisis dan rancangan berkaitan dengan studi literatur pada bab sebelumnya. Oleh karena itu, bab ini akan terdiri dari analisis permasalahan dan rancangan solusi.

\section{Analisis Permasalahan}

	\subsection{Analisis Permasalahan Analisis Sentimen}
	Pada penelitian ini, digunakan dataset yang sama dengan dataset \parencite{FarhanKhodra2017}. Dataset tersebut merupakan ulasan-ulasan dari situs TripAdvisor yang kemudian dilabeli positif untuk nilai 3-5 dan negatif untuk nilai 1-2. Setiap ulasan, yang terdiri dari satu sampai tiga kalimat, dapat memiliki frasa sentimen di awal, di tengah, dan di akhir dokumen. Berikut contoh ulasan yang berlabel negatif.

	“Kemaren sengaja coba makan di marugame udon karena istri suka banget sama udon. jadi pesan lah yang kuah kaldu ayam, dan anak saya minta gorengan kroket. saya kira kuahnya langsung dari pancinya harusnya panas, tapi ternyata cuman hangat, yah kecewa juga sih, gorengannya juga sama udah dingin” 

	Dapat dilihat dari paragraf di atas, kalimat awal pada paragraf tersebut tidak mengandung sentimen apapun. Frasa sentimen baru ditemui di akhir paragraf. Berikut contoh ulasan yang berlabel positif.

	“berlokasi di pusat jakarta.. sangat mudah mengakses ke berbagai tempat strategis dan keramaian. hotel sangat bersih, pelayanan ramah dan makanan sesuai lidah indonesia atau western. sangat direkomendasikan.”

	Dapat dilihat dari paragraf di atas, frasa sentimen positif sudah langsung diutarakan pada awal paragraf. Tetapi tidak semua ulasan memiliki satu sentimen saja. Berikut ini adalah salah satu contoh ulasan pada dataset yang mengandung frasa sentimen negatif dan positif disaat bersamaan. 

	“Kelihatan dari luar memang menarik tempatnya. Suasananya juga cozy. Saat makan siang juga ramai pengunjung. AC tidak berasa, jadi lumayan panas di dalam. Memesan nasi dengan irisan slice pork belly, nasinya tanpa bumbu apa- apa jadi kering banget, irisan pork belly nya dominan lemak dan minyak. Kita dapat 6 slice, tapi sesudah makan slice ke 3, rasanya tidak sanggup lagi karena merasa eneg dengan lemak pork belly nya.”

	Dapat dilihat dari paragraf di atas, tiga kalimat pertama menunjukkan opini positif sedangkan 3 kalimat terakhir menunjukkan opini negatif. Tetapi berdasarkan analisis keseluruhan, dapat dilihat bahwa ulasan tersebut bersentimen negatif.

	Penelitian terkait oleh \parencite{FarhanKhodra2017} dan \parencite{CrisdayantiPurwarianti2019} sudah mengekplorasi berbagai jenis representasi dokumen dan topologi untuk menyelesaikan analisis sentimen level dokumen pada data tersebut. Meski sudah mendapatkan hasil, penelitian terkait sebelumnya belum mengeksplorasi penggunaan \textit{language model} dalam bentuk apapun, terlebih lagi penggunaan \textit{transfer learning} lintas bahasa.

	Perkembangan dalam bidang teknik pemrosesan teks dewasa ini telah memunculkan berbagai jenis \textit{language model} dan teknik pembangunannya. Salah satunya adalah \textit{language model} lintas bahasa bernama XLM oleh \parencite{LampleConneau2019} yang dibangun tanpa menggunakan korpus paralel atau kamus kata antar bahasa. Hal ini memungkinkan kita menggunakan representasi antar bahasanya untuk melatih model analisis sentimen bahasa Indonesia dengan dataset bahasa lainnya.

	\subsection{Analisis Permasalahan Klasifikasi Teks Ujaran Kebencian \& Kasar}


% Data adalah inti dari pembangunan sistem analisis sentimen. Sedikitnya data yang sudah dianotasi pada bahasa Indonesia menyulitkan pengembangan sistem analisis sentimen yang lebih akurat. Di lain hal, bahasa asing seperti bahasa Inggris memiliki jauh lebih banyak data yang sudah dianotasi. Teknik yang dapat memanfaatkan data dari bahasa asing untuk membangun sistem pemrosesan teks di bahasa Indonesia diperlukan.

% Teknik pemrosesan teks memungkinkan pembangunan ruang embedding antar bahasa. Dengan memanfaatkan representasi teks antar bahasa, dimungkinkan melatih classifier pada suatu bahasa dan menggunakannya pada bahasa lain seperti pada \parencite{Artetxe_Schwenk_2019}.

% Tidak hanya itu, gap antar domain yang menjadi ulasan pun dapat diselesaikan dengan teknik augmentasi yang \parencite{Lai_Oguz_Yang_Stoyanov_2019} kembangkan. 

\section{Rancangan Solusi}
	Berdasarkan analisis permasalahan yang dijabarkan pada bab sebelumnya, kita dapat menggunakan \textit{language model} antar bahasa XLM yang sudah dilatih dengan data dari 100 bahasa untuk memperkaya data analisis sentimen bahasa Indonesia. Diantara 100 bahasa yang sudah dilatih pada \textit{language model} XLM adalah bahasa Indonesia dan bahasa Inggris. Sub-bab selanjutnya akan membahas pemilihan dataset bahasa Inggris yang akan digunakan dan rancangan eksperimen untuk membuktikan tujuan dari tugas akhir ini.

	\subsection{Komponen Dataset Sumber Analisis Sentimen}
	Pada tugas akhir ini, dipilih dataset \textit{Yelp review}\footnote{\url{https://www.yelp.com/dataset/challenge}} dikarenakan jumlahnya dan kesesuaian domainnya dengan dataset TripAdvisor. Dataset polaritas Yelp reviews  dibangun dengan mengubah semua rating bintang 1 dan 2 menjadi negatif, dan rating bintang 3 dan 4 menjadi positif. Rincian dari dataset dapat dilihat pada tabel \ref{tab:detail_yelp_review}.

	\begin{table}[ht]
	    \centering
	    \caption{Rincian banyak data pada tiap partisi dan label di dataset \textit{Yelp review}.}
	    \begin{tabular}{@{}cc|cc@{}}
	    \multicolumn{1}{c}{} &\multicolumn{1}{c}{} &\multicolumn{2}{c}{Partisi} \\ 
	    \multicolumn{1}{c}{} & 
	    \multicolumn{1}{c|}{} & 
	    \multicolumn{1}{c}{\textit{Train}} & 
	    \multicolumn{1}{c}{\textit{Test}} \\ 
	    \cline{2-4}
	    \multirow[c]{2}{*}{\rotatebox[origin=tr]{90}{Label}}
	    & Positif  & 280.000 & 19.000   \\[1.5ex]
	    & Negatif  & 280.000 & 19.000   \\ 
	    \cline{2-4}
	    \end{tabular}
	    \label{tab:detail_yelp_review}
	\end{table}

	Sesuainya domain antara dataset sumber dan target sangat penting dikarenakan perbedaan domain dapat menurunkan performa klasifikasi \parencite{Lai_Oguz_Yang_Stoyanov_2019}. Hal ini dikarenakan sentimen tidak diutarakan dengan cara yang sama pada domain yang berbeda. Ulasan pada domain produk di toko \textit{e-commerce} akan berbeda dengan ulasan pada domain restoran. Pada domain produk di toko \textit{e-commerce}, ulasan akan berkutat pada aspek seperti kecepatan pengiriman barang, keselamatan barang, dan kualitas dari barang yang dibeli. Sedangkan ulasan pada domain restoran akan berkutat pada keramahan pelayanan, suasana restoran, dan kualitas makanan.

	\subsection{Komponen Dataset Bahasa Inggris Analisis Sentimen}
	
	\subsection{Komponen Dataset Bahasa Inggris Klasifikasi Ujaran Kebencian \& Kasar}

	\subsection{Komponen Klasifikasi}
	Skema eksperimen akan mengikuti eksperimen yang dilakukan oleh \parencite{LampleConneau2019} pada masalah klasifikasi antar bahasa dengan dataset XNLI \parencite{Conneau_Rinott_Lample_Williams_Bowman_Schwenk_Stoyanov_2018}. Pada eksperimennya, model yang sudah dilatih dengan data Wikipedia 100 bahasa ini di-\textit{finetune} dengan dataset tolak ukur klasifikasi. Lebih tepatnya, ditambahkan \textit{classifier} linier di atas \textit{hidden state}pertama dari \textit{language model} yang sudah di-\textit{pretrained}, yang kemudian semua parameternya dilatih pada dataset pelatihan NLI bahasa Inggris. Skema yang sama akan digunakan dengan sumber bahasa Inggris dan target bahasa Indonesia. Seluruh ekperimen akan dikembangkan menggunakan PyTorch \parencite{paszke2017automatic}.

	Untuk membuktikan tujuan tugas akhir, dikembangkan 3 eksperimen dengan detail sebagai berikut:
	\begin{enumerate}
		\item \textbf{Tipe A}\\
		Pada \textit{baseline} ini, model XLM yang sudah dilatih pada data Wikipedia 100 bahasa di-\textit{finetune} dengan data bahasa Indonesia saja. Model ini kemudian digunakan untuk melakukan klasifikasi sentimen bahasa Indonesia. Detail ilustrasi dapat dilihat pada Gambar \ref{fig:ilustrasi_solusi_1}.

		\begin{figure}[h]
		    \centering
		    \includegraphics[width=1\textwidth]{resources/Arsitektur-TA-1.png}
		    \caption{ Ilustrasi \textit{finetuning} \textit{baseline} A.}
		    \label{fig:ilustrasi_solusi_1}
		\end{figure}

		\item \textbf{Tipe B}\\
		Pada tipe B ini, dilakukan ekperimen \textit{zero-shot learning}. Model XLM yang sudah dilatih pada data Wikipedia 100 bahasa hanya di-\textit{finetune} dengan data bahasa Inggris saja. Model ini kemudian langsung digunakan untuk melakukan klasifikasi sentimen bahasa Indonesia. Detail ilustrasi dapat dilihat pada Gambar \ref{fig:ilustrasi_solusi_2}.

		\begin{figure}[h]
		    \centering
		    \includegraphics[width=1\textwidth]{resources/Arsitektur-TA-2.png}
		    \caption{ Ilustrasi \textit{finetuning} B.}
		    \label{fig:ilustrasi_solusi_2}
		\end{figure}

		\item \textbf{Tipe C}\\
		Pada tipe C ini, dilakukan ekperimen \textit{transfer learning} antar bahasa. Model XLM yang sudah dilatih pada data Wikipedia 100 bahasa di-\textit{finetune} dengan data bahasa Inggris. Model ini kemudian di-\textit{finetune} kembali dengan data bahasa Indonesia sebelum digunakan untuk melakukan klasifikasi sentimen bahasa Indonesia. Detail ilustrasi dapat dilihat pada Gambar \ref{fig:ilustrasi_solusi_3}.

		\begin{figure}[h]
		    \centering
		    \includegraphics[width=1\textwidth]{resources/Arsitektur-TA-3.png}
		    \caption{ Ilustrasi \textit{finetuning} C.}
		    \label{fig:ilustrasi_solusi_3}
		\end{figure}
		
	\end{enumerate}

	Dalam setiap eksperimen, teks dipraproses terlebih dahulu menggunakan \textit{shared vocabulary} yang dibuat melalui \textit{Byte Pair Encoding (BPE)} hasil penelitian \parencite{LampleConneau2019}. Di setiap eksperimen juga akan digunakan \textit{callback} berupa \textit{EarlyStopping} dan \textit{ReduceLROnPlateau}. Penggunaan \textit{callback} \textit{EarlyStopping} digunakan agar model tidak \textit{overfit}. Sedangkan \textit{callback} \textit{ReduceLROnPlateau} digunakan untuk membantu model mencapai performa yang lebih baik.
    \chapter{Eksperimen dan Pembangunan Sistem}

\section{Lingkungan Eksperimen dan Pembangunan Sistem}

Eksperimen dilakukan di platform \href{https://www.kaggle.com}{Kaggle} dengan Kernel gratisnya. Bahasa Pemrograman yang digunakan adalah bahasa Python versi 3.6.6. Pada \textit{fine-tuning} layer terakhir digunakan CPU 1xSingle core hyper threaded Xeon Processors @2.3Ghz, 46MB Cache. Pada \textit{fine-tuning} penuh digunakan TPU v3-8. Eksperimen dikembangkan menggunakan kombinasi pustaka PyTorch \parencite{paszke2017automatic} dan TensorFlow \parencite{tensorflow2015}.

Untuk memastikan bahwasanya eksperimen \textit{reproducible} dan tidak memiliki faktor acak, semua opsi \textit{seed} baik di System, Python, hingga pustaka Pytorch ditetapkan agar hasil tidak berubah jika eksperimen dijalankan berkali-kali dengan kondisi yang sama.

\begin{lstlisting}[language=Python]
def set_seed():
    seed=1
    random.seed(seed)
    torch.manual_seed(seed)
    torch.cuda.manual_seed_all(seed)
    np.random.seed(seed)
    os.environ['PYTHONHASHSEED'] = str(seed)
    torch.backends.cudnn.deterministic = True
\end{lstlisting}


\section{Eksperimen}
Eksperimen akan dilakukan dengan 2 model multilingual (XLM-R \& Multilingual BERT). Pada tiap eksperimen akan dilakukan variasi total data (500 / 1000 / 2500 / 5000 / 7500 / MAX) dan kelipatan bahasa asing pada skenario 3. Untuk konfigurasi model, callback berupa EarlyStopping dan ReduceLROnPlateau. Penggunaan callback EarlyStopping digunakanagar model tidak overfit. Sedangkan callback ReduceLROnPlateau digunakan untuk membantu model mencapai performa yang lebih baik. Semua eksperimen dijalankan hingga diberhentikan oleh callback EarlyStopping.


    \subsection{Hasil fine-tune layer terakhir}
        Berikut hasil eksperimen dengan model XLM-R: 
        \begin{enumerate}
            \item Analisis sentimen dengan dataset B \\
            \begin{figure}[ht]
                \centering
                \includegraphics[width=0.75\textwidth]{resources/plot-head-prosa-xlmr.png}
                \caption{Plot dataset B dengan model XLM-R.}
                \label{fig:plot_head_prosa_xlmr}
            \end{figure}
            Dapat dilihat pada Gambar \ref{fig:plot_head_prosa_xlmr}, performa analisis sentimen dataset B sangat terbantu dengan penambahan data berbahasa Inggris. Di berbagai total data, penambahan bahasa Inggris menambah rata-rata F1-score sebesar 0.229. Penambahan performa terbesar diobservasi pada eksperimen dengan jumlah data yang sedikit. Performa model meningkat dari 0.734 pada skenario monolingual hingga mencapai 0.819 pada penambahan bahasa Inggris sebanya 7 kali lipat dari total bahasa Indonesianya.

            \item Analisis sentimen dengan data dataset A Advisor \\
            \begin{figure}[ht]
                \centering
                \includegraphics[width=0.75\textwidth]{resources/plot-head-trip-xlmr.png}
                \caption{Plot dataset A dengan model XLM-R.}
                \label{fig:plot_head_trip_xlmr}
            \end{figure}
            Dapat dilihat pada Gambar \ref{fig:plot_head_trip_xlmr}, performa analisis sentimen dataset A terbantu dengan penambahan data berbahasa Inggris. Di berbagai total data, penambahan bahasa Inggris menambah rata-rata F1-score sebesar 0.107. Penambahan performa terbesar juga diobservasi pada eksperimen dengan jumlah data yang sedikit. Performa model meningkat dari 0.794 pada skenario monolingual hingga mencapai 0.823 pada penambahan bahasa Inggris sebanya 8 kali lipat dari total bahasa Indonesianya.

            \item Klasifikasi ujaran kebencian \& kasar \\
            \begin{figure}[ht]
                \centering
                \includegraphics[width=0.75\textwidth]{resources/plot-head-toxic-xlmr.png}
                \caption{Plot klasifikasi ujaran kebencian dengan model XLM-R.}
                \label{fig:plot_head_toxic_xlmr}
            \end{figure}
            Dapat dilihat pada Gambar \ref{fig:plot_head_toxic_xlmr}, performa klasifikasi ujaran kebencian tidak terlalu terbantu dengan penambahan data berbahasa Inggris. Meski performa model sempat terbantu, penambahan lebih banyak data bahasa Inggris menurunkan performa model.

        \end{enumerate}
        
        Melalui rangkaian eksperimen dengan model XLM-R di atas, dapat dilihat korelasi antara perbedaan performa skenario \textit{zero-shot} dan \textit{monolingual} dengan performa skenario \textit{multilingual learning}. Eksperimen sentimen analisis dataset B memiliki rata-rata perbedaan skenario 1 dan 2 sebesar 0.0005. Dengan perbedaan yang sangat kecil, penambahan bahasa Inggris sangat bermanfaat. Eksperimen sentimen analisis dataset A memiliki rata-rata perbedaan skenario 1 dan 2 sebesar 0.044. Dengan perbedaan yang cukup besar, penambahan bahasa Inggris menjadi berkurang manfaatnya. Terakhir, eksperimen klasifikasi ujaran kebencian memiliki perbedaan skenario 1 dan 2 sebesar 0.093. Dengan perbedaan yang sangat besar, penambahan bahasa Inggris menjadi kurang bermanfaat dan bahkan menurunkan performa model.

        Berikut hasil eksperimen dengan model mBERT: 

        \begin{figure}[ht]
            \minipage{0.5\textwidth}
              \includegraphics[width=\linewidth]{resources/plot-head-prosa-mbert.png}
              \caption{Plot dataset B model mBERT}\label{fig:plot_head_prosa_mbert}
            \endminipage\hfill
            \minipage{0.5\textwidth}
              \includegraphics[width=\linewidth]{resources/plot-head-trip-mbert.png}
              \caption{Plot dataset A model mBERT.}\label{fig:plot_head_trip_mbert}
            \endminipage
        \end{figure}

        \begin{figure}[ht]
            \centering
            \includegraphics[width=0.5\textwidth]{resources/plot-head-toxic-mbert.png}
            \caption{Plot ujaran kebencian model mBERT.}\label{fig:plot_head_toxic_mbert}
        \end{figure}

        Melalui rangkaian eksperimen dengan model mBERT di atas, dapat dilihat hal yang sama dengan eksperimen model XLM-R. Secara umum performa mBERT lebih lemah dibanding XLM-R. Selain itu, dapat dilihat juga biasnya \textit{pretraining} mBERT ke bahasa Inggris. Di berbagai total data, performa \textit{zero-shot} lebih bagus dibanding pasangan \textit{monolingual}-nya. Hal ini dikarenakan dalam pelatihan mBERT memiliki jumlah data bahasa Inggris yang tidak proporsional dengan data lainnya. Berbeda dengan XLM-R yang melakukan \textit{sampling} pada tiap \textit{batch} pelatihannya untuk memastikan performa model proporsional antar bahasa.
            

    \subsection{Hasil fine-tune penuh}
        Eksperimen fine-tune penuh masih sedang dijalankan. Berikut hasil eksperimen yang sudah selesai dengan model XLM-R pada analisis sentimen dataset B dan klasifikasi ujaran kebencian :

        \begin{figure}[ht]
            \minipage{0.5\textwidth}
              \includegraphics[width=\linewidth]{resources/plot-full-prosa-xlmr.png}
              \caption{Plot dataset A model XLM-R}\label{fig:plot_full_prosa_xlmr}
            \endminipage\hfill
            \minipage{0.5\textwidth}
              \includegraphics[width=\linewidth]{resources/plot-full-trip-advisor-xlmr.png}
              \caption{Plot dataset B model XLM-R}\label{fig:plot_full_prosa_xlmr}
            \endminipage
        \end{figure}

        \begin{figure}[ht]
            \minipage{0.5\textwidth}
                \includegraphics[width=\linewidth]{resources/plot-full-toxic-xlmr.png}
                \caption{Plot Ujaran kebencian model XLM-R.}\label{fig:plot_full_toxic_xlmr}
            \endminipage\hfill
            \minipage{0.5\textwidth}
                \includegraphics[width=\linewidth]{resources/plot-full-trip-advisor-xlmr-duplicate.png}
                \caption{Plot dataset A model XLM-R duplikat dihilangkan}\label{fig:plot_fuLL_trip_duplicate}
            \endminipage
        \end{figure}
        Melalui sedikit eksperimen dengan model XLM-R, dapat dilihat performa model XLM-R sangat bagus. Pada analisis sentimen dataset B model berhasil memprediksi seluruh data tes dengan sempurna di skenario 3 dengan kelipatan bahasa Inggris 1.5. Pada klasifikasi ujaran kebencian model berhasil mencapai F1-Score 0.892 di skenario 3 dengan kelipatan bahasa Inggris 1. 

        Meski begitu, pada analisis sentimen dataset A, model yang dilatih dengan bahasa Indonesia dan campuran memiliki performa yang jauh lebih jelek dibanding model yang dilatih dengan bahasa Inggris saja. Selain itu, hasil fine-tune penuh klasifikasi ujaran kebencian tidak dapat dibandingkan dengan penelitian sebelumnya secara langsung. Dua hal ini akan dibahas pada bab selanjutnya

\subsection{Analisis hasil}
    Sub bab ini akan membahas lebih detail hasil yang didapatkan pada bab sebelumnya

    \subsection{Hasil fine-tune penuh klasifikasi ujaran kebencian}
    Untuk dapat membandingkan secara langsung penggunaan \textit{multilingual language model}, pelatihan dengan konfigurasi yang sama dengan skenario 2 penelitian \parencite{Ibrohim_Budi_2019} dilakukan. Hasilnya dapat dilihat pada Tabel \ref{tab:toxic_xlm_r_comparable}.

    \begin{table}[]
        \centering
        \begin{tabular}{|r|l|l|l|}
        \hline
        \multicolumn{1}{|l|}{} & \textbf{Hate Speech} & \textbf{Abusive} & \textbf{Average}      \\ \hline
        \textbf{Accuracy}      & 85.573273\%          & 93.470008\%      & \textbf{89.5216405\%} \\ \hline
        \textbf{F1}            & 0.85331737           & 0.93094791       & \textbf{0.89213264}   \\ \hline
        \end{tabular}
        \caption{Hasil fine-tune penuh klasifikasi ujaran kebencian \textit{comparable}.}
        \label{tab:toxic_xlm_r_comparable}
    \end{table}

    \subsection{Hasil fine-tune penuh analisis sentimen data A}
    Setelah ditelusuri, ternyata dataset A memiliki 2573 teks duplikat. Sehingga, dari total 12389 data teks, hanya 9816 teks yang unik. Hal ini menyebabkan \textit{language model} overfit kepada data ini dan tidak mempelajari permasalahan secara sepenuhnya. Eksperimen fine-tune penuh analisis sentimen data A dijalankan kembali pada dataset yang dihilangkan teks duplikatnya. Hasilnya dapat dilihat pada Gambar \ref{fig:plot_fuLL_trip_duplicate}

    



    
        

    \chapter{Kesimpulan dan Saran}

\section{Kesimpulan}
Berikut beberapa kesimpulan yang dapat ditarik dari tugas akhir ini:
\begin{enumerate}
    \item Penambahan dataset bahasa Inggris dapat membantu meningkatkan performa klasifikasi teks bahasa Indonesia menggunakan \textit{multilingual language model}. Hanya saja, disparitas antara data bahasa Indonesia dengan data bahasa Inggris, yang dapat dilihat dari performa \textit{zero-shot} dan \textit{monolingual baseline}-nya, harus diperhatikan. Pada kasus analisis sentimen dengan dataset Prosa dan Trip Advisor, jarak tersebut masing-masing rata-rata 0.0005 dan 0.044. Sehingga eksperimen sentimen analisis skenario 3 pada data tersebut mendapat peningkatan F1-score masing-masing rata-rata 0.229 dan 0.107. Sedangkan pada klasifikasi ujaran kebencian yang jaraknya 0.093, penambahan data bahasa Inggris menjadi kurang bermanfaat dan bahkan menurunkan performa model.
    \item Penggunaan \textit{multilingual language model} yang di \textit{fine-tune} sepenuhnya sangat mangkus dalam klasifikasi teks bahasa Indonesia. Pada eksperimen sentimen analisis dataset Prosa, model ini mendapatkan F1-score sempurna. Sebuah peningkatan absolut dari penelitian sebelumnya yang mendapatkan F1-score 0,9369. Pada eksperimen klasifikasi ujaran kebencian, model ini mendapatkan F1-score 0,892 dan akurasi 89.4\%. Penelitian sebelumnya yang menggunakan 3 label, bukan yang disimplifikasi menjadi 2 seperti di penelitian ini, mendapatkan average accuracy tertinggi 77.36\%. Meski tidak dapat dibandingkan langsung dengan eksperimen terdekat penelitian sebelumnya, peningkatan ini tetap signifikan. 
    
    (\textbf{Pertanyaan Bu:} Saya kan simplifikasi jadi 2 label dikarenakan alasan yang saya sebutkan di bab III.1.2 dan III.2.2. Yaitu karena dataset bahasa Inggrisnya cuma punya 2 label. Selanjutnya apa saya coba saja melatih modelnya dengan bahasa Indonesia saja tapi 3 labelnya? Supaya bisa dibandingkan performanya)

\end{enumerate}

\section{Saran}
Berikut beberapa saran yang dapat ditarik dari tugas akhir ini:
\begin{enumerate}
    \item Terdapatnya disparitas antara dataset dapat menyebabkan turunnya performa. Untuk penelitian selanjutnya, dapat dicoba beberapa cara untuk mengatasi hal ini. Beberapa diantaranya adalah seperti penelitian \parencite{Lai_Oguz_Yang_Stoyanov_2019} yang menggunakan \textit{universal data augmentation} untuk mengurangi jarak tadi.
\end{enumerate}
    %----------------------------------------------------------------%

    % Daftar pustaka
    \printbibliography[title={Daftar Pustaka}]

    % Index
    \appendix

    \addcontentsline{toc}{part}{Lampiran}
    \part*{Lampiran}

    \chapter{Algoritma \textit{Byte Pair Encoding} Sederhana}
\label{appendix:simple_bpe_algorithm}

Algoritma (contoh nama file: \(bpe.py\)):
\begin{lstlisting}[language=Python]
import re, collections
def get_stats(vocab):
    pairs = collections.defaultdict(int)
    for word, freq in vocab.items():
        symbols = word.split()
        for i in range(len(symbols)-1):
            pairs[symbols[i],symbols[i+1]] += freq
    return pairs

def merge_vocab(pair, v_in):
    v_out = {}
    bigram = re.escape(' '.join(pair))
    p = re.compile(r'(?<!\S)' + bigram + r'(?!\S)')
    for word in v_in:
        w_out = p.sub(''.join(pair), word)
        v_out[w_out] = v_in[word]
    return v_out

vocab = {'l o w </w>' : 5, 'l o w e r </w>' : 2,
'n e w e s t </w>':6, 'w i d e s t </w>':3}
vocab_test = {'l o w e s t </w>': 1}

num_merges = 10
for i in range(num_merges):
    pairs = get_stats(vocab)
    best = max(pairs, key=pairs.get)
    print('~~~')
    vocab = merge_vocab(best, vocab)
    vocab_test = merge_vocab(best, vocab_test)
    print("best: ", best)
    print("vocab: ", vocab)
    print("vocab_test: ", vocab_test)
\end{lstlisting}

Setelah dijalankan di mesin bersistem operasi Ubuntu 18.04 dengan perintah
\begin{lstlisting}[language=bash]
    $ python3 bpe.py
\end{lstlisting}

akan didapatkan keluaran sebagai berikut:
\begin{lstlisting}[language=bash]
~~~
best:  ('e', 's')
vocab:  {'l o w </w>': 5, 'l o w e r </w>': 2, 'n e w es t </w>': 6, 'w i d es t </w>': 3}
vocab_test:  {'l o w es t </w>': 1}
~~~
best:  ('es', 't')
vocab:  {'l o w </w>': 5, 'l o w e r </w>': 2, 'n e w est </w>': 6, 'w i d est </w>': 3}
vocab_test:  {'l o w est </w>': 1}
~~~
best:  ('est', '</w>')
vocab:  {'l o w </w>': 5, 'l o w e r </w>': 2, 'n e w est</w>': 6, 'w i d est</w>': 3}
vocab_test:  {'l o w est</w>': 1}
~~~
best:  ('l', 'o')
vocab:  {'lo w </w>': 5, 'lo w e r </w>': 2, 'n e w est</w>': 6, 'w i d est</w>': 3}
vocab_test:  {'lo w est</w>': 1}
~~~
best:  ('lo', 'w')
vocab:  {'low </w>': 5, 'low e r </w>': 2, 'n e w est</w>': 6, 'w i d est</w>': 3}
vocab_test:  {'low est</w>': 1}
~~~
best:  ('n', 'e')
vocab:  {'low </w>': 5, 'low e r </w>': 2, 'ne w est</w>': 6, 'w i d est</w>': 3}
vocab_test:  {'low est</w>': 1}
~~~
best:  ('ne', 'w')
vocab:  {'low </w>': 5, 'low e r </w>': 2, 'new est</w>': 6, 'w i d est</w>': 3}
vocab_test:  {'low est</w>': 1}
~~~
best:  ('new', 'est</w>')
vocab:  {'low </w>': 5, 'low e r </w>': 2, 'newest</w>': 6, 'w i d est</w>': 3}
vocab_test:  {'low est</w>': 1}
~~~
best:  ('low', '</w>')
vocab:  {'low</w>': 5, 'low e r </w>': 2, 'newest</w>': 6, 'w i d est</w>': 3}
vocab_test:  {'low est</w>': 1}
~~~
best:  ('w', 'i')
vocab:  {'low</w>': 5, 'low e r </w>': 2, 'newest</w>': 6, 'wi d est</w>': 3}
vocab_test:  {'low est</w>': 1}
\end{lstlisting}
    % \input{chapters/appendix-2}

\end{document}
